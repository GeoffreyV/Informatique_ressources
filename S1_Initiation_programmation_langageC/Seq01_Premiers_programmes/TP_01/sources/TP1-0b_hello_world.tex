
\section{Votre premier programme : Hello World}

\subsection{Hello World en C}

\begin{UPSTIManipulation}{Hello World}
	Trois étapes sont nécessaires pour créer un Hello World en langage C :
	\begin{enumerate}
		\item Inclure la bibliothèque standard d'entrée/sortie (\texttt{stdio.h}). -> Ecrire \texttt{\#include <stdio.h>} en haut du fichier
		\item Définir la fonction principale \texttt{main}. -> \texttt{int main(void) { ... }}
		\item Utiliser la fonction \texttt{printf} à l'intérieur de la fonction main pour afficher le message.
		      \begin{lstlisting}[language=c]
printf("Hello, World!\n");
\end{lstlisting}
		\item Ajouter \texttt{return 0;} à la fin pour indiquer que le programme est terminé.
	\end{enumerate}
\end{UPSTIManipulation}

Votre programme devra donc ressembler à ça :
\begin{lstlisting}[language=c]
#include <stdio.h>

int main(void)
{
    printf("Hello, World!\n");
    return 0;
}
\end{lstlisting}

\begin{UPSTIinfor}{Besoin d'aide ?}
	CS50 Duck Debugger est là pour vous aider !
	Cliquer sur l'icône en forme de canard dans la colonne de gauche pour ouvrir l'extension.
	Vous pouvez lui demander "En Français" à la fin de votre prompt pour obtenir de l'aide en français.  S'il vous dit qu'il ne parle qu'anglais, insistez, il finit souvent par abdiquer.
\end{UPSTIinfor}



\begin{UPSTIManipulation}{Compiler et exécuter le programme}
	\begin{itemize}
		\item[$\Box$] Dans le terminal, assurez-vous d'être dans le dossier \texttt{TP1/0\_hello} (où se trouve votre fichier \texttt{hello.c} -- sinon, utilisez la commande \texttt{cd} pour naviguer jusqu'à ce dossier).
		\item[$\Box$] Compiler le programme avec la commande \texttt{make hello}.
		\item[$\Box$] Exécuter le programme compilé avec la commande \texttt{./hello}. Vous devriez voir "Hello, World!" s'afficher dans le terminal.
	\end{itemize}
	\lstinputlisting[language=bash, style=console]{console/69XbpUL0KsVsH9V9FqBNHuNUf.txt}
	\begin{itemize}
		\item Si vous voyez le message, félicitations ! Vous venez d'écrire   votre premier programme en C. \textbf{Dans le cas contraire, demandez de l'aide à l'enseignant, à vos camarades ou au CS50 Duck Debugger.}
	\end{itemize}
\end{UPSTIManipulation}