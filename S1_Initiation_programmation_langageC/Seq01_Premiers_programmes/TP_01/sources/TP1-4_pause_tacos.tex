
\section{La pause Tacos}
\subsection{Niveau 1}

\begin{UPSTIcahierDesCharges}{Pause Tacos - Niveau 1}
	Votre programme tient un Tacos dont vous devez inventer le nom.
	Il reçoit un client, qui lui commande un certain nombre de tacos et de kebab et votre programme lui dit annonce alors le montant total de la commande
	\begin{itemize}
		\item[$\Box$] Le programme Affiche d'abord une phrase de bienvenue
		      \begin{lstlisting}[language=bash,style=console]
Bonjour, bienvenu chez <nom_du_tacos>
\end{lstlisting}
		\item[$\Box$] Le programme demande alors le nombre de Tacos, puis le nombre de Kebab voulu
		\item[$\Box$] Le programme écrit alors le prix total de la commande, arrondi au centième.
		\item[$\Box$] Le programme écrit ensuite le message de remerciement
		      \begin{lstlisting}[language=bash,style=console]
Montant total : <montant_total> euros
Merci pour votre commande chez <nom_du_tacos>
\end{lstlisting}
	\end{itemize}
	\begin{center}
		\begin{tabular}{|l|l|}
			\hline
			Produit & Prix   \\
			\hline
			Tacos   & 6,30 € \\
			Kebab   & 5,50 € \\
			\hline
		\end{tabular}
	\end{center}
	\hrule
	\textbf{Checks:}
	\begin{lstlisting}[language=bash,style=console]
check50 IUT-GEII-Annecy/exercices/2025/info1/tp1/3_tacos/niveau1
\end{lstlisting}
\end{UPSTIcahierDesCharges}


\begin{UPSTIinfor}{Exemple de sortie pour cet exercice}
	\lstinputlisting[language=bash, style=console]{console/hD1h6Kxu9xtn1pfwgW2e7oetj.txt}
\end{UPSTIinfor}

\begin{UPSTIinfor}{Fiche méthode}
	Pour ce programme, comme pour beaucoup, posez-vous ces questions, dans l'ordre.
	\begin{enumerate}
		\item "Si j'étais à la place du programme, comment je ferai ?"
		      \begin{itemize}
			      \item Vous pourrez en déduire les étapes de l'algorithme
		      \end{itemize}
		\item "Si j'étais vraiment très mauvais en mémorisation, qu'est-ce que je devrais écrire ?"
		      \begin{itemize}
			      \item Vous pourrez en déduire les variables du programme
		      \end{itemize}
		      \begin{itemize}
			      \item[$\Box$] Lesquelles sont des entiers \texttt{int} ?
			      \item[$\Box$] Lesquelles sont des nombres à virgule \texttt{float} ?
			      \item[$\Box$] Autres ?
		      \end{itemize}
		\item "Y a-t-il des 'cas bizarre' ?"
	\end{enumerate}
\end{UPSTIinfor}

\subsection{Niveau 2}
Même chose avec gestion des stocks

\begin{UPSTIcahierDesCharges}{Pause Tacos - Niveau 2 - Gestion des stocks}
	\begin{itemize}
		\item[$\Box$] Même cahier des charges que le niveau 1 auquel s'ajoute :
		\item[$\Box$] Le restaurant a un stock limité.
		      \begin{itemize}
			      \item[$\Box$] Après avoir demandé le nombre de tacos et de kebab :
			            \begin{itemize}
				            \item[$\Box$] Si le client demande unnombre négatif de l'un, l'autre ou les deux :
				                  \begin{itemize}
					                  \item[$\Box$] affiche le message
					                        \begin{lstlisting}[language=bash,style=console]
			ERREUR : Valeurs négatives interdites.
\end{lstlisting}
					                  \item[$\Box$] Arrête le programme sans autre affichage -> \texttt{return 1;}
				                  \end{itemize}
				            \item[$\Box$] Si les stocks sont insuffisant, le programme
				                  \begin{itemize}
					                  \item[$\Box$] affiche un message selon le tableau ci-dessous
					                  \item[$\Box$] affiche ensuite le message de remerciement du Niveau 1
				                  \end{itemize}
				            \item[$\Box$] Sinon, pas de changement par rapport au niveau 1
			            \end{itemize}
		      \end{itemize}
	\end{itemize}

	\begin{center}
		\begin{tabular}{|l|l|}
			\hline
			Produit hors stock & Sortie attendue                                               \\
			\hline
			Kebab              & \texttt{Désolé, nous n'avons pas assez de Kebab}              \\
			Tacos              & \texttt{Désolé, nous n'avons pas assez de Tacos}              \\
			Tacos et Kebab     & \texttt{Désolé, nous n'avons pas assez de Tacos, ni de Kebab} \\
			\hline
		\end{tabular}
	\end{center}
	\hrule
	\begin{lstlisting}[language=bash,style=console]
check50 IUT-GEII-Annecy/exercices/2025/info1/tp1/3_tacos/niveau2
\end{lstlisting}
\end{UPSTIcahierDesCharges}

\subsection{Bonus : Niveau 3}

\begin{UPSTIcahierDesCharges}{Pause Tacos - Niveau 3 - Réduction}
	\begin{itemize}
		\item[$\Box$] Même cahier des charges que précédemment
		\item[$\Box$] Si le client commande plus de 5 articles au total, une réduction de 10\% est appliquée sur le montant total de la commande
		\item[$\Box$] Les sorties attendues sont identiques aux cahiers des charges précédents
	\end{itemize}
	\hrule
	\begin{lstlisting}[language=bash,style=console]
check50 IUT-GEII-Annecy/exercices/2025/info1/tp1/3_tacos/niveau3
\end{lstlisting}
\end{UPSTIcahierDesCharges}