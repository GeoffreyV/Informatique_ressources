
\section{Êtes-vous majeur ?}
Exerçons-nous avec quelques conditions simple

\begin{UPSTIManipulation}{Majeur ou mineur}
	\begin{itemize}
		\item[$\Box$] aller dans le dossier \texttt{tp1/3\_age} puis ouvrir age.c
		      \begin{lstlisting}[language=bash,style=console]
cd
cd tp1/3_age
code age.c
\end{lstlisting}
		\item[$\Box$] Compléter le programme pour les Niveaux 1 et 2
	\end{itemize}
\end{UPSTIManipulation}

\subsection{Niveau 1}

\begin{UPSTIcahierDesCharges}{Age - Niveau 1}
	\begin{itemize}
		\item[$\Box$] Le programme demande quel age a l'utilisateur
		\item[$\Box$] Le programme répond alors selon la logique suivante :
		      \begin{itemize}
			      \item[$\Box$] Sinon, si l'utilisateur a strictement moins de 18 ans : \texttt{Vous êtes un mineur.}
			      \item[$\Box$] Sinon \texttt{Vous êtes un adulte.}
		      \end{itemize}
	\end{itemize}
	\hrule
	\textbf{Checks:}
	\begin{lstlisting}[language=bash,style=console]
check50 IUT-GEII-Annecy/exercices/2025/info1/tp1/3_age/niveau1
\end{lstlisting}
\end{UPSTIcahierDesCharges}

\subsection{Niveau 2}

\begin{UPSTIcahierDesCharges}{Age - Niveau 2}
	\begin{itemize}
		\item[$\Box$] Le programme demande quel age a l'utilisateur
		\item[$\Box$] Le programme répond alors selon la logique suivante :
		      \begin{itemize}
			      \item[$\Box$] Si l'utilisateur a strictement moins de 12 ans : \texttt{Vous êtes un enfant.}
			      \item[$\Box$] Sinon, si l'utilisateur a strictement moins de 18 ans : \texttt{Vous êtes un mineur.}
			      \item[$\Box$] Sinon, si l'utilisateur a strictement moins de 60 ans : \texttt{Vous êtes un adulte.}
			      \item[$\Box$] Sinon, si l'utilisateur a strictement moins de 120 ans : \texttt{Vous êtes un sénior.}
			      \item[$\Box$] Sinon : \texttt{Vous êtes un menteur.}
		      \end{itemize}
	\end{itemize}
	\hrule
	\textbf{Checks:}
	\begin{lstlisting}[language=bash,style=console]
check50 IUT-GEII-Annecy/exercices/2025/info1/tp1/3_age/niveau2
\end{lstlisting}
\end{UPSTIcahierDesCharges}