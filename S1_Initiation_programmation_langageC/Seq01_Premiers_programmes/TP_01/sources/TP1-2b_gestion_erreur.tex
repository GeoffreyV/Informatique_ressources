\section{Nos premières conditions !}

Si vous donnez des valeurs négatives en entrée des programmes précédents, ils vous répondront une surface négative. Testez-le !

Tout le monde sait qu'une surface est toujours positive... Alors interdisons au programme de répondre n'importe quoi

\begin{UPSTIcahierDesCharges}{Valeurs négatives interdites}
	\begin{itemize}
		\item Même cahier des charges que précédemment sauf que :
		\item Après que l'utilisateur a rentré \textbf{toutes les longueurs} :
		      \begin{itemize}
			      \item Si l'une ou l'autre des valeur est négative, le programme répondra
			            \begin{lstlisting}[language=bash,style=console]
		ERREUR : Valeurs negatives interdites.
\end{lstlisting}
		      \end{itemize}
	\end{itemize}
	\hrule
	\textbf{Checks :}
	\begin{lstlisting}[language=bash,style=console]
IUT-GEII-Annecy/exercices/2025/info1/tp1/2_geometrie/cercle/niveau2
IUT-GEII-Annecy/exercices/2025/info1/tp1/2_geometrie/rectangle/niveau2
IUT-GEII-Annecy/exercices/2025/info1/tp1/2_geometrie/geometrie/all
\end{lstlisting}
\end{UPSTIcahierDesCharges}



\begin{UPSTIManipulation}{Valeurs négatives impossibles}
	\begin{itemize}
		\item[$\Box$] Compléter les programmes \textit{cercle.c} et \textit{rectangle.c} pour remplir le cahier des charges.
		\item[$\Box$] Tester vos programmes à la main
	\end{itemize}
\end{UPSTIManipulation}