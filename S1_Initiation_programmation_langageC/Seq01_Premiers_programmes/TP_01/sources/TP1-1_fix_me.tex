\section{Fix Me : Cherchez l'erreur}

Oups ! Le fichier fixme.c contient plein d'erreurs qui rendent la compilation impossible.
A vous de les corriger !


\begin{UPSTIManipulation}{Corrige le fichier}
	\begin{itemize}
		\item[$\Box$] Se rendre dans le dossier \texttt{1\_fixme}
		      \begin{itemize}
			      \item[$\Box$] Soit en revenant d'abord à la racine avec \texttt{cd} seul
			            \begin{lstlisting}[language=bash,style=console]
	cd
	cd tp1/1_fixme
\end{lstlisting}
			      \item[$\Box$] Soit directement depuis \texttt{0\_hello} :
			            \begin{lstlisting}[language=bash,style=console]
	cd ../1_fixme
\end{lstlisting}
			      \item[$\Box$] Dans les deux cas, votre terminal doit maintenant commencer par
			            \begin{lstlisting}[language=bash,style=console]
	tp1/1_fixme $
\end{lstlisting}
		      \end{itemize}
		\item[$\Box$] Tenter de compiler \texttt{make fixme} -> Des erreurs apparaissent
		\item[$\Box$] Corriger le fichier \texttt{fixme.c}
		\item[$\Box$] Compiler, puis tester votre code
		      \begin{lstlisting}[language=bash,style=console]
	make fixme
	./fixme
\end{lstlisting}
		\item[$\Box$] \textbf{Si tout fonctionne}, exécuter
		      \begin{lstlisting}[language=bash,style=console]
check50 iut-geii-annecy/exercices/2025/info1/tp1/1_fixme/fixme
\end{lstlisting}
	\end{itemize}
\end{UPSTIManipulation}