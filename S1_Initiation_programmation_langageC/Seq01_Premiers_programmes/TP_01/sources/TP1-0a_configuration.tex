
\section{Environnement de Trvail}

\subsection{Première configuration}

\begin{UPSTIManipulation}{Créer un compte GitHub}
	\begin{itemize}
		\item[$\Box$] Si vous n'avez pas encore de compte GitHub, créez-en un :
		      \begin{itemize}
			      \item \textbf{Nom d'utilisateur :} \texttt{<p>-<nom>} avec p, la première lettre de votre prénom, puis votre nom.
			            \begin{itemize}
				            \item Exemple : j-doe pour John Doe
			            \end{itemize}
			      \item \textbf{Adresse Mail :} \texttt{<adresse email académique>}
			      \item[$\Box$] Lien pour inscription : \href{https://github.com}{GitHub.com}.
		      \end{itemize}
		\item[$\Box$] Rejoignez le cours du semestre 1, en cliquant sur le lien correspondant à votre groupe.
	\end{itemize}
	\begin{center}
		\begin{tabular}{|l|l|}
			\hline
			Groupe & Lien                                                                             \\
			\hline
			A      & \href{https://submit.cs50.io/invites/9a6f6de5408b4022baebeb336f409261}{Groupe A} \\
			B      & \href{https://submit.cs50.io/invites/f70e5bcd71c941a7acd2cbd7739218d8}{Groupe B} \\
			C      & \href{https://submit.cs50.io/invites/88bf318c8b054f7a959ea7371f2b8011}{Groupe C} \\
			D      & \href{https://submit.cs50.io/invites/e58130b1d7874ca481da62a0795da120}{Groupe D} \\
			\hline
		\end{tabular}
	\end{center}
	\begin{itemize}
		\item[$\Box$] Lancez l'IDE, pour cela :
		      \begin{itemize}
			      \item[$\Box$] Rendez-vous sur la page de \href{https://cs50.dev}{cs50.dev}.
			      \item[$\Box$] Cliquez sur \texttt{Log in}. \textit{La première ouverture peut prendre quelques minutes}
		      \end{itemize}
	\end{itemize}
\end{UPSTIManipulation}

\begin{UPSTIinfor}{GitHub ? Qu'est-ce que c'est ?}
	GitHub est une plateforme de développement collaboratif qui permet de stocker et de partager du code.
	Elle permet notamment le versionning \textit{(Pouvoir revenir à toutes les versions précédentes du code)}, le suivi des modifications et facilite grandement la collaboration.
	C'est un outil incontournable dans le monde professionnel du développement logiciel.
	Pour ce premier semestre, l'utilisation de GitHub est invisible pour vous. Elle se fait via les commande de cs50
\end{UPSTIinfor}

\subsection{Le terminal}

\subsubsection{Présentation}

En bas de la fenêtre de vsCode, vous trouverez un terminal. C'est un outil qui vous permet d'interagir avec votre système d'exploitation en utilisant des commandes textuelles.

\begin{UPSTIinfor}{Liste des commandes utiles}
	Voici quelques commandes de base que vous utiliserez fréquemment :
	\begin{itemize}
		\item \texttt{clear} : nettoie l'affichage dans le terminal
	\end{itemize}
	Les éléments entre chevrons <>, sont à remplacer en fonction du contexte
	\begin{itemize}
		\item \texttt{cd <nom\_du\_dossier>} : change le répertoire courant pour \texttt{<nom\_du\_dossier>}.
		      \begin{itemize}
			      \item \texttt{cd ..} permet de remonter dans le dossier parent
		      \end{itemize}
		\item \texttt{code <nom\_du\_fichier>} permet d'ouvrir le fichier \texttt{<nom\_du\_fichier>} en le créant si besoin
		\item \texttt{cp <source> <destination>} : copie un fichier ou dossier de \texttt{<source>} à \texttt{<destination>}.
		\item \texttt{ls} : liste les fichiers et dossiers dans le répertoire courant.
		\item \texttt{mv <source> <destination>} : déplace ou renomme un fichier ou dossier de \texttt{<source>} à \texttt{<destination>}.
		\item \texttt{mkdir <nom\_du\_dossier>} : crée un nouveau dossier nommé \texttt{<nom\_du\_dossier>}.
		\item \texttt{rm <nom\_du\_fichier>} : supprime le fichier nommé \texttt{<nom\_du\_fichier>}.
		\item \texttt{rmdir <nom\_du\_dossier>} : supprime un dossier nommé \texttt{<nom\_du\_dossier>} (le dossier doit être vide).
	\end{itemize}
\end{UPSTIinfor}

\begin{UPSTIinfor}{L'autocomplétion dans le terminal}
	Bien souvent, le terminal peut deviner ce que vous voulez faire à partir du début de la commande, à l'aide de la touche \texttt{Tab}.
\end{UPSTIinfor}

\begin{UPSTIManipulation}{On s'exerce !}
	\begin{itemize}
		\item[$\Box$] Exécuter les lignes suivantes, \textbf{les unes après les autres}.
		      \begin{itemize}
			      \item[$\Box$] Pour chacune, écrire ce que fait la commande.
			            \begin{lstlisting}[language=bash,style=console]
mkdir monPremierDossier
ls
cd monPremierDossier
mkdir sous-dossier1 sous-dossier2
ls
cd ..
ls
cd monPremierDossier/sous-dossier1
cd ../sous-dossier2
cd ..
mv sous-dossier1 nouveau-nom
ls
cd
rm monPremierDossier
rm -r monPremierdossier
\end{lstlisting}
		      \end{itemize}
	\end{itemize}
	A vous de jouer !!
	\begin{itemize}
		\item[$\Box$] Créer un dossier portant votre nom, à l'intérieur duquel vous créerez deux dossiers : un à votre prenom, l'autre tp1
		\item[$\Box$] Faire vérifier par l'enseignant
	\end{itemize}
\end{UPSTIManipulation}

\begin{UPSTIwarning}{Informations importantes à propos du terminal :}
	\begin{itemize}
		\item Certains raccourcis ne sont pas les même qu'habituellement
		      \begin{itemize}
			      \item \texttt{Ctrl-Maj-C} pour copier
			      \item \texttt{Maj-INSER} pour coller
		      \end{itemize}
		\item Les commandes ne répondent souvent rien lorsqu'elle réussissent
		\item Il est impossible de bouger le curseur avec la souris. Il faut utiliser les flèches
	\end{itemize}
\end{UPSTIwarning}

\subsection{Téléchargement du dossier de TP}

Prêts ? Allons-y ! Récupérons les fichiers pour commencer à coder !

\begin{UPSTIManipulation}{Premières commandes dans le terminal}
	\lstinputlisting[language=bash, style=console]{console/qiLs8ysmvZjWn8MoMIZMel1wZ.txt}
	\begin{itemize}
		\item[$\Box$] Télécharger les fichiers du tp1 :
		      \begin{itemize}
			      \item[$\Box$] Copier la commande suivante puis la coller la dans le terminal \texttt{Maj-INSER}. Appuyer sur Entrée
			            \begin{lstlisting}[language=bash,style=console]
wget https://github.com/IUT-GEII-Annecy/squelettes/releases/download/branch-2025/tp1.zip
\end{lstlisting}
		      \end{itemize}
		\item[$\Box$] Décompresser le dossier et supprimer le \textit{.zip}
		      \begin{itemize}
			      \item[$\Box$] \textbf{(Essayer l'autocomplétion en tapant \texttt{unzip t} puis en appuyant sur \texttt{Tab})} :
			            \begin{lstlisting}[language=bash,style=console]
unzip tp1.zip
\end{lstlisting}
			      \item[$\Box$] Supprimer le fichier \texttt{tp1.zip} :
			            \begin{lstlisting}[language=bash,style=console]
rm tp1.zip
\end{lstlisting}
			      \item[$\Box$] Répondre \texttt{y} s'il demande confirmation
		      \end{itemize}
		\item[$\Box$] Vous pouvez afficher la liste des fichiers et dossiers avec la commande \texttt{ls}.
		      \begin{lstlisting}[language=bash,style=console]
ls
ls tp1/
\end{lstlisting}
		\item[$\Box$] Naviguer dans le dossier \texttt{tp1/0\_hello}
		      \begin{lstlisting}[language=bash,style=console]
cd tp1/0_hello
\end{lstlisting}
		\item[$\Box$] Créer un fichier \texttt{hello.c} :
		      \begin{lstlisting}[language=bash,style=console]
code hello.c
\end{lstlisting}
		      \begin{itemize}
			      \item[$\Box$] Un nouvel onglet devrait s'ouvrir dans vsCode avec un fichier vide nommé \texttt{hello.c}.
			      \item \textit{Vous pouvez également créer un fichier en cliquant sur l'icône "Nouveau fichier" dans l'explorateur de fichiers à gauche.}
		      \end{itemize}
	\end{itemize}
\end{UPSTIManipulation}