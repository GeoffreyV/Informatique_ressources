% ==================== NIVEAU 2 — VARIABLES / TYPES (10) ====================

\element{niveau2}{
\begin{question}{n2-priorite-ops}
Quelle est la valeur finale de \lstinline|r| ?
\begin{lstlisting}[language=C]
int r = 10 - 2 * 3 + 8 / 2;
\end{lstlisting}
\begin{reponses}
  \bonne{8}
  \mauvaise{22}
  \mauvaise{-1}
  \mauvaise{4}
  \mauvaise{10}
  \mauvaise{12}
\end{reponses}
\end{question}
}

\element{niveau2}{
\begin{question}{n2-types-cast}
Quelle est la valeur de ces expressions ?
\begin{lstlisting}[language=C]
int a=13, b=5;
(float)a/b;
\end{lstlisting}
\begin{reponses}
  \bonne{2.6}
  \mauvaise{2}
  \mauvaise{2.5}
  \mauvaise{13/5}
  \mauvaise{3}
  \mauvaise{Erreur}
\end{reponses}
\end{question}
}

\element{niveau2}{
\begin{question}{n2-open-declaration}
Déclarer et initialiser des variables adaptées pour : masse d’un objet (kg, décimales), nombre de pièces (entier), code article (texte).
\evaluationProfLignes{1}{3}
\end{question}
}

\element{niveau2}{
\begin{question}{n2-open-trace}
Prédire les valeurs finales de \lstinline|x|, \lstinline|y|, \lstinline|z|.
\begin{lstlisting}[language=C]
int x=3,y=1,z=2;
x=y+z;
z=x-y;
y=x+z;
\end{lstlisting}
\evaluationProfLignes{2}{2}
\end{question}
}

\element{niveau2}{
\begin{question}{n2-conv-mix}
Quel est le résultat de \lstinline|(int)(9.0/4*10)| ?
\begin{reponses}
  \bonne{22}
  \mauvaise{23}
  \mauvaise{2}
  \mauvaise{2.25}
  \mauvaise{9}
  \mauvaise{Erreur}
\end{reponses}
\end{question}
}

\element{niveau2}{
\begin{question}{n2-open-init}
Écrire une initialisation pour : booléen \texttt{isRegistered}, réel \texttt{moyenne}, entier \texttt{nbAbsences}.
\evaluationProfLignes{2}{2}
\end{question}
}

\element{niveau2}{
\begin{question}{n2-affichage-melange}
Que va afficher ce code ?
\begin{lstlisting}[language=C]
int n=9, d=4;
float p=9, q=4;
n = n - d/2;
p = p / q;
printf("%d %f\n", n, p);
\end{lstlisting}
\begin{reponses}
  \bonne{7 ; 2.25}
  \mauvaise{7 ; 2}
  \mauvaise{9 ; 2.25}
  \mauvaise{7 ; 9/4}
  \mauvaise{Erreur}
  \mauvaise{8 ; 2.25}
\end{reponses}
\end{question}
}

\element{niveau2}{
\begin{question}{n2-open-type-justif}
Pour : tension électrique (12.5 V), nombre d’interrupteurs (3), état marche/arrêt, proposer un type et justifier.
\evaluationProfLignes{3}{3}
\end{question}
}

\element{niveau2}{
\begin{question}{n2-modulo}
Que vaut \lstinline|18 % 5| ?
\begin{reponses}
  \bonne{3}
  \mauvaise{2}
  \mauvaise{0}
  \mauvaise{1}
  \mauvaise{4}
  \mauvaise{Erreur}
\end{reponses}
\end{question}
}

\element{niveau2}{
\begin{question}{n2-open-erreurs}
Pourquoi la somme répétée de 0.1 en \lstinline|float| peut-elle ne pas donner exactement 1.0 ?  
Expliquer en deux phrases.
\evaluationProfLignes{2}{2}
\end{question}
}

% ==================== NIVEAU 2 — CONDITIONS (10) ====================

\element{niveau2}{
\begin{question}{n2-cond-majorite}
On veut distinguer : \og Enfant \fg{} si <12, \og Adolescent \fg{} si 12–17, \og Majeur \fg{} si ≥18.  
Quelle structure est correcte ?
\begin{reponses}
  \bonne{\lstinline[language=C]|if(age<12)... else if(age<18)... else...|}
  \mauvaise{\lstinline[language=C]|if(age<12)... if(age<18)... else...|}
  \mauvaise{\lstinline[language=C]|if(age<18)... else if(age<12)... else...|}
  \mauvaise{Impossible en C}
  \mauvaise{Toujours afficher "Majeur"}
  \mauvaise{Toujours afficher "Mineur"}
\end{reponses}
\end{question}
}

\element{niveau2}{
\begin{question}{n2-cond-intervalle}
Condition correcte pour tester \(0 < t < 100\) :
\begin{reponses}
  \bonne{\lstinline[language=C]|t>0 && t<100|}
  \mauvaise{\lstinline[language=C]|t>0 || t<100|}
  \mauvaise{\lstinline[language=C]|t>=0 && t<=100|}
  \mauvaise{\lstinline[language=C]|t=0 && t=100|}
  \mauvaise{\lstinline[language=C]|t<0 && t>100|}
  \mauvaise{Impossible}
\end{reponses}
\end{question}
}

\element{niveau2}{
\begin{question}{n2-open-acces}
On veut : Accès si \texttt{isRegistered} et \texttt{hasBadge}, ou si \texttt{isAdmin}.  
Écrire la condition en C.
\evaluationProfLignes{2}{2}
\end{question}
}

\element{niveau2}{
\begin{question}{n2-bool-eval1}
Avec a=3, b=7, c=7, que vaut \lstinline|(a<b)&&(b==c)| ?
\begin{reponses}
  \bonne{true}
  \mauvaise{false}
  \mauvaise{Erreur}
  \mauvaise{0}
  \mauvaise{1}\bareme{0.5}
  \mauvaise{Inconnu}
\end{reponses}
\end{question}
}

\element{niveau2}{
\begin{question}{n2-bool-eval2}
Avec a=3, b=7, c=7, que vaut \lstinline|(a>=b)||(c!=7)| ?
\begin{reponses}
  \bonne{false}
  \mauvaise{true}
  \mauvaise{Erreur}
  \mauvaise{0}\bareme{0.5}
  \mauvaise{1}
  \mauvaise{Inconnu}
\end{reponses}
\end{question}
}

\element{niveau2}{
\begin{question}{n2-open-traduction}
Traduire en condition C :  
\og On accorde une réduction si l’étudiant a une moyenne ≥14 et est assidu \fg{}.
\evaluationProfLignes{1}{2}
\end{question}
}

\element{niveau2}{
\begin{question}{n2-cond-syntaxe}
Laquelle est une écriture correcte ?
\begin{reponses}
  \bonne{\lstinline[language=C]|if(x>=0 && x<=10) printf("OK");|}
  \mauvaise{\lstinline[language=C]|if(x>=0 && x<=10) { printf("OK") }|}
  \mauvaise{\lstinline[language=C]|if x>=0 && x<=10 then printf("OK");|}
  \mauvaise{\lstinline[language=C]|if(x>=0 && x<=10); printf("OK");|}
  \mauvaise{\lstinline[language=C]|if(x>=0) and (x<=10) printf("OK");|}
  \mauvaise{Aucune}
\end{reponses}
\end{question}
}

\element{niveau2}{
\begin{question}{n2-open-cinema}
Rédiger une condition pour afficher le tarif :  
<12 ans = 4€, 12–25 ans = 6€, >25 ans = 9€.
\evaluationProfLignes{1}{3}
\end{question}
}

\element{niveau2}{
\begin{question}{n2-bool-eval3}
Avec a=4, b=7, que vaut \lstinline|!(a%2)&&(b%2)| ?
\begin{reponses}
  \bonne{vrai}
  \mauvaise{faux}
  \mauvaise{Erreur}
  \mauvaise{0}
  \mauvaise{1}
  \mauvaise{Inconnu}
\end{reponses}
\end{question}
}

\element{niveau2}{
\begin{question}{n2-open-exemple}
Donner deux exemples concrets de conditions logiques dans la vie quotidienne et les traduire en pseudo-code.
\evaluationProfLignes{3}{3}
\end{question}
}
