\documentclass[a4paper]{article}
\usepackage[box,completemulti]{automultiplechoice}
\usepackage{csvsimple}
\usepackage[T1]{fontenc}
\usepackage[utf8]{inputenc}
\usepackage[french]{babel}
\usepackage{ifthen}

\begin{document}

% ----- Chargement des étudiants (id,name,level) -----
\csvreader[head to column names]{students.csv}{}{% Début de la boucle sur les étudiants
  \def\STUDENTID{\id}%
  \def\STUDENTNAME{\name}%
  \def\LEVEL{\level}%
  % Graine de randomisation stable par étudiant
  \setrandomseed{\STUDENTID}%

  \onecopy{1}{% une copie par étudiant
    \begin{center}
      \Large \textbf{QCM — Variables \& Conditions}\\[2mm]
      \normalsize Niveau \LEVEL\quad|\quad \STUDENTNAME\ (ID \STUDENTID)
    \end{center}

    % --- Bloc d'identification AMC (code élève pour liaison) ---
    \AMCcodeGridInt{5}{\STUDENTID}

    % Questions communes à tous
    \shufflegroup{common}
    \insertgroup{common}

    % Sélection selon le niveau
    \ifthenelse{\equal{\LEVEL}{1}}{%
      \shufflegroup{L1}\insertgroup{L1}%
    }{}
    \ifthenelse{\equal{\LEVEL}{2}}{%
      \shufflegroup{L2}\insertgroup{L2}%
    }{}
    \ifthenelse{\equal{\LEVEL}{3}}{%
      \shufflegroup{L3}\insertgroup{L3}%
    }{}

    \clearpage
  }% fin \onecopy
}% fin boucle CSV

\end{document}
