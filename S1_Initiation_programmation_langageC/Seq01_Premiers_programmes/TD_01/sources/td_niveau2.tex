\section{Niveau 2}

\subsection{Opérations et conversions}

\begin{UPSTIexercice}{Opérations et priorités}
Soit le code suivant : 
\begin{lstlisting}[language=C]
int r = 10 - 2 * 3 + 8 / 2;
int s = r + 5 * (3 - 1);
\end{lstlisting}
\UPSTIquestion{Donner l'ordre exact des opérations.}
\UPSTIquestion{Donner les valeurs finales de \verb|r| et \verb|s|}
\end{UPSTIexercice}

\noindent\begin{minipage}{.49\linewidth}
\begin{UPSTIexercice}{Conversions et cast}
On considère :
\begin{lstlisting}[language=C]
int a = 13, b = 5;
float u = 13, v = 5;
int A = a/b;
float C = (float)a / b;
int D = (float)a / b;
int E = a%b;
\end{lstlisting}
\UPSTIquestion{Donner les valeurs des variables A, B, C, D et E.}
\end{UPSTIexercice}
\end{minipage}\hfill
\begin{minipage}{.49\linewidth}
    \begin{UPSTIinfor}{Le transtypage (cast)}
        En langage C, on peut momentanément considérer qu'une variable est d'un certain type au lieu de celui déclaré. Il suffit de spécifier le type entre parenthèses devant la variable. \\
        Par exemple : 
        \begin{lstlisting}[language=c]
int variable_entiere;
float resultat = (float) variable_entiere/2;\end{lstlisting}
        Durant cette opération, \verb|variable_entiere| sera considérée comme \verb|float|.
    \end{UPSTIinfor}
\end{minipage}


\begin{UPSTIexercice}{Prédire l'affichage (mélange de types)}
\begin{lstlisting}[language=C]
int n = 9, d = 4;
float p = 9, q = 4;
n = n - d/2;
p = p / q;
printf("A:%d\n", n);
printf("B:%f\n", p);
printf("C:%f\n", (float)n / d);
printf("D:%d\n", n % d);
\end{lstlisting}
\UPSTIquestion{Prédire chaque ligne affichée et expliquer le rôle de la \textbf{division entière} dans la mise à jour de \verb|n|.}
\end{UPSTIexercice}

\subsection{Conditions multiples et booléens}

\begin{UPSTIexercice}{Intervalle inclusif et exclusif}
\UPSTIquestion{Ecrire un condition en langage C pour tester \(5 \le x \le 10\).}
\UPSTIquestion{Ecrire ce même test d'une façon différente.}
\UPSTIquestion{Expliquer pourquoi \lstinline[language=c]{(x > 5 || x < 10)} est incorrecte.}
\UPSTIquestion{Expliquer pourquoi \lstinline[language=c]{(5 <= x <= 10)} est incorrecte.}
\end{UPSTIexercice}


\begin{UPSTIexercice}{Autorisation d'accès}
On considère trois variables booléennes (le type \textbf{bool} est défini dans la bibliothèque \texttt{stdbool}) :
\begin{lstlisting}[language=c]
    bool isRegistered;
    bool hasBadge; 
    bool isAdmin;
\end{lstlisting}

On veut accorder l'accès à une personne si elle possède un badge et est enregistrée. L'accès sera aussi accordé si la personne est administratrice.
\UPSTIquestion{Donner une condition booléenne correctement parenthésée qui accorde, ou non, l'accès.}
\end{UPSTIexercice}


\begin{UPSTIexercice}{Calcul d'expressions booléennes}
Soient \verb|a=3, b=7, c=7|.
\UPSTIquestion{Évaluer les expressions booléennes suivante (pour chacune, dire si elle renvoie \textbf{vrai} ou \textbf{faux}) : }
\begin{multicols}{3}
\begin{itemize}
    \item \verb|(a<b) && (b==c)|
    \item \verb$(a>=b) || (c!=7)$
    \item \verb|!(a<b) && (b==c)|
    \item \verb|(a<b) && !(b==c)|
    \item \verb$(a<b) || (b==c)$
    \item \verb$!(a<b) || (b==c)$
\end{itemize}
\end{multicols}
\end{UPSTIexercice}


