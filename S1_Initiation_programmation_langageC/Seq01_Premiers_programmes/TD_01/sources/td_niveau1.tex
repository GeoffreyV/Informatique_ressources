\section{Niveau 1}
\subsection{Variables}
\begin{UPSTIexercice}{Informations personnelles (partie 1)}
    On souhaite stocker des informations simples concernant un étudiant :
    \begin{itemize}
        \item Son prénom
        \item Son âge
        \item Sa taille (en centimètres)
    \end{itemize}
    \UPSTIquestion{Pour chaque variable, proposer un type de données adapté et justifier le choix.}
    \UPSTIquestion{Écrire les lignes de \textbf{déclaration} de ces variables (sans initialisation).}
    \UPSTIquestion{Ecrire des lignes \textbf{d'initialisation} cohérentes pour un étudiant fictif.}
\end{UPSTIexercice}

\begin{UPSTIexercice}{Affectations successives}
    On considère le programme suivant :
    \begin{lstlisting}[language=C]
int x = 2;
int y = 5;
int z;
z = x + y;
y = z - 1;
x = y + z + x;
\end{lstlisting}
    \UPSTIquestion{Donner les valeurs finales de \verb|x|, \verb|y| et \verb|z|.}
\end{UPSTIexercice}


\begin{UPSTIexercice}{Types adaptés}
On considère les informations suivantes :
\begin{itemize}
    \item Nombre d'étudiants d'une promo
    \item Température extérieure
    \item Numéro de salle
    \item Prix d'un sandwich
\end{itemize}

\UPSTIquestion{Pour chacune, donner le type le plus approprié et la déclaration de la variable.}
\end{UPSTIexercice}

\begin{UPSTIexercice}{Prédire l'affichage (entier vs réel)}
On exécute le programme suivant.
\begin{lstlisting}[language=C]
int a = 7, b = 2;
float x = 7, y = 2;
printf("A:%d\n", a / b);
printf("B:%f\n", x / y);
printf("C:%f\n", a / (float)b);
printf("D:%d\n", a % b);
\end{lstlisting}

\UPSTIquestion{Si ce n'est pas déjà fait, lire l'encart sur la division, page~\pageref{info:division}}
\UPSTIquestion{Prédire l'affichage généré par les lignes ci-dessus.}
\end{UPSTIexercice}



\subsection{Conditions}

\begin{UPSTIexercice}{Majorité (partie 1)}
On veut afficher \og Majeur \fg{} si \verb|age >= 18|, sinon \og Mineur \fg{}.
\UPSTIquestion{Décrire la logique en français ou pseudo-code avec un \texttt{if/else}. Donner deux exemples d'entrées et la sortie attendue.}
\end{UPSTIexercice}

\begin{UPSTIexercice}{Qui est le plus grand ?}
    On souhaite écrire un programme qui affiche lequel de deux nombres \verb|a| et \verb|b| est le plus grand. S'ils sont égaux, le programme le dira. 
\UPSTIquestion{Proposer les lignes de codes qui, pour deux entiers \verb|a| et \verb|b|, affichent le plus grand ou s'ils sont égaux.}
\end{UPSTIexercice}


\begin{UPSTIexercice}{Nombre mystère (logique)}
    On considère un programme dont deux variables ont été déclarées : \verb|nombre_mystere| et \verb|nombre_joueur|. Le nombre mystère n'est pas connu du joueur. Il doit le deviner en tentant des valeurs au hasard. 
\begin{lstlisting}
int nombre_mystere = 7; 
int nombre_joueur = get_int("Deviner le nombre mystere : ");
\end{lstlisting}
On veut écrire \textbf{bravo} si l'utilisateur à trouvé le nombre, \textbf{Raté} sinon.
\UPSTIquestion{Ecrire un pseudo-code décrivant la logique à implémenter}
\UPSTIquestion{Traduire ces lignes en langage C}
\UPSTIquestion{Réécrire le programme pour afficher si le nombre proposé est \textbf{Trop petit} et \textbf{Trop grand}.}
\end{UPSTIexercice}
