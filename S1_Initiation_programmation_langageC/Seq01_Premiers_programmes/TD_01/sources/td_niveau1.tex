\section{Niveau 1}
\subsection{Variables}
\begin{UPSTIexercice}{Informations personnelles}
    On souhaite stocker des informations simples concernant un étudiant :
    \begin{itemize}
        \item Son prénom
        \item Son âge
        \item Sa taille (en mètres)
    \end{itemize}
    \UPSTIquestion{Pour chaque variable, proposer un type de données adapté et justifier le choix.}
    \UPSTIquestion{Écrire les lignes de \textbf{déclaration} de ces variables (sans initialisation).}
    \UPSTIquestion{Ecrire des lignes \textbf{d'initialisation} cohérentes pour un étudiant fictif.}
\end{UPSTIexercice}

\begin{UPSTIprofOnlyEnv}%
    \begin{UPSTIcorrectionP}{Informations personnelles}
        \begin{lstlisting}[language=c]
string prenom;
int age;
float taille;
prenom = "John";
age = 20;
taille = 1.74;
        \end{lstlisting}
    \end{UPSTIcorrectionP}
\end{UPSTIprofOnlyEnv}

\begin{UPSTIexercice}{Affectations successives}
    On considère le programme suivant :
    \begin{lstlisting}[language=C]
int x = 2;
int y = 5;
int z;
z = x + y;
y = z - 1;
x = y + z + x;
    \end{lstlisting}
    \UPSTIquestion{Donner les valeurs finales de \verb|x|, \verb|y| et \verb|z|.}
\end{UPSTIexercice}

\begin{UPSTIprofOnlyEnv}
    \begin{UPSTIcorrectionP}{Affectations successives}
        \begin{lstlisting}[language=C]
// Etapes de calcul :
int x = 2;      // x = 2
int y = 5;      // y = 5
int z;          // z = ?
z = x + y;      // z = 2 + 5 = 7
y = z - 1;      // y = 7 - 1 = 6
x = y + z + x;  // x = 6 + 7 + 2 = 15

// Valeurs finales :
x = 15;
y = 6;
z = 7;
        \end{lstlisting}
    \end{UPSTIcorrectionP}
\end{UPSTIprofOnlyEnv}


\begin{UPSTIexercice}{Types adaptés}
    On considère les informations suivantes :
    \begin{itemize}
        \item Nombre d'étudiants d'une promo
        \item Température extérieure
        \item Numéro de salle
        \item Prix d'un sandwich
    \end{itemize}

    \UPSTIquestion{Pour chacune, donner le type le plus approprié et la déclaration de la variable.}
\end{UPSTIexercice}

\begin{UPSTIprofOnlyEnv}
    \begin{UPSTIcorrectionP}{Types adaptés}
        \begin{lstlisting}[language=C]
// Nombre d'etudiants d'une promo : entier naturel
int nb_etudiants;

// Temperature exterieure : reel
float temperature;

// Numero de salle : souvent alphanumerique -> texte
string numero_salle;

// Prix d'un sandwich : reel en euros, préférer le double pour la précision (cf Niveau 3)
// (Remarque : on peut aussi stocker en centimes avec un int)
double prix_sandwich;\end{lstlisting}
    \end{UPSTIcorrectionP}
\end{UPSTIprofOnlyEnv}


\begin{UPSTIexercice}{Prédire l'affichage (entier vs réel)}
    On exécute le programme suivant.
    \begin{lstlisting}[language=C]
int a = 7, b = 2;
float x = 7, y = 2;
printf("A:%d\n", a / b);
printf("B:%f\n", x / y);
printf("C:%f\n", a / (float)b);
printf("D:%d\n", a % b);
\end{lstlisting}

    \UPSTIquestion{Si ce n'est pas déjà fait, lire l'encart sur la division, page~\pageref{info:division}}
    \UPSTIquestion{Prédire l'affichage généré par les lignes ci-dessus.}
\end{UPSTIexercice}

\begin{UPSTIprofOnlyEnv}
    \begin{UPSTIcorrectionP}{Prédire l'affichage (entier vs réel)}
        \begin{itemize}
            \item a / b est une division entière (entiers) : 7/2 = 3
            \item x / y est une division réelle (flottants) : 7.0/2.0 = 3.5
            \item a / (float)b force une division réelle (cast) : 3.5
            \item a \% b est le reste de la division euclidienne : 7 \% 2 = 1
        \end{itemize}
        Affichage attendu :
        \begin{lstlisting}[language=bash]
A:3
B:3.500000
C:3.500000
D:1
\end{lstlisting}
    \end{UPSTIcorrectionP}
\end{UPSTIprofOnlyEnv}



\subsection{Conditions}

\begin{UPSTIexercice}{Majorité (partie 1)}
    On veut afficher \og Majeur \fg{} si \verb|age >= 18|, sinon \og Mineur \fg{}.
    \UPSTIquestion{Décrire la logique en français ou pseudo-code avec un \texttt{if/else}. Donner deux exemples d'entrées et la sortie attendue.}
\end{UPSTIexercice}

\begin{UPSTIprofOnlyEnv}
    \begin{UPSTIcorrectionP}{Majorité (partie 1)}
        \begin{lstlisting}[language=C]
// Pseudo-code :
// si age >= 18 alors afficher "Majeur"
// sinon afficher "Mineur"

// Traduction C (extrait) :
if (age >= 18)
{
    printf("Majeur\n");
}
else
{
    printf("Mineur\n");
}

// Exemples :
// age = 17 -> "Mineur"
// age = 21 -> "Majeur"
        \end{lstlisting}
    \end{UPSTIcorrectionP}
\end{UPSTIprofOnlyEnv}


\begin{UPSTIexercice}{Qui est le plus grand ?}
    On souhaite écrire un programme qui affiche lequel de deux nombres \verb|a| et \verb|b| est le plus grand. S'ils sont égaux, le programme le dira.
    \UPSTIquestion{Proposer les lignes de codes qui, pour deux entiers \verb|a| et \verb|b|, affichent le plus grand ou s'ils sont égaux.}
\end{UPSTIexercice}

\begin{UPSTIprofOnlyEnv}
    \begin{UPSTIcorrectionP}{Qui est le plus grand ?}
        \begin{lstlisting}[language=C]
if (a > b)
{
    printf("a est plus grand (%d)\n", a);
}
else if (b > a)
{
    printf("b est plus grand (%d)\n", b);
}
else
{
    printf("a et b sont egaux (%d)\n", a);
}
        \end{lstlisting}
    \end{UPSTIcorrectionP}
\end{UPSTIprofOnlyEnv}


\begin{UPSTIexercice}{Nombre mystère (logique)}
    On considère un programme dont deux variables ont été déclarées : \verb|nombre_mystere| et \verb|nombre_joueur|. Le nombre mystère n'est pas connu du joueur. Il doit le deviner en tentant des valeurs au hasard.
    \begin{lstlisting}
int nombre_mystere = 7; 
int nombre_joueur = get_int("Deviner le nombre mystere : ");
    \end{lstlisting}
    On veut écrire \textbf{bravo} si l'utilisateur à trouvé le nombre, \textbf{Raté} sinon.
    \UPSTIquestion{Ecrire un pseudo-code décrivant la logique à implémenter}
    \UPSTIquestion{Traduire ces lignes en langage C}
    \UPSTIquestion{Réécrire le programme pour afficher si le nombre proposé est \textbf{Trop petit} et \textbf{Trop grand}.}
\end{UPSTIexercice}

\begin{UPSTIprofOnlyEnv}
    \begin{UPSTIcorrectionP}{Nombre mystère (logique)}
        \begin{lstlisting}[language=C]
// 1) Pseudo-code :
// si nombre_joueur == nombre_mystere afficher "Bravo"
// sinon afficher "Rate"

// 2) Traduction C (version simple) :
if (nombre_joueur == nombre_mystere)
{
    printf("Bravo\n");
}
else
{
    printf("Rate\n");
}

// 3) Variante avec "Trop petit" / "Trop grand" :
if (nombre_joueur == nombre_mystere)
{
    printf("Bravo\n");
}
else if (nombre_joueur < nombre_mystere)
{
    printf("Trop petit\n");
}
else
{
    printf("Trop grand\n");
}
        \end{lstlisting}
    \end{UPSTIcorrectionP}
\end{UPSTIprofOnlyEnv}
