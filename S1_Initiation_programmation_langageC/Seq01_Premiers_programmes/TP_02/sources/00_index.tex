
\section{Introdution}
Une structure de contrôle permet de modifier le flux d'exécution d'un programme en fonction de conditions ou de répétitions.
Nous avons déjà vu les structures conditionnelles \texttt{if}, \texttt{else if} et \texttt{else} dans le TP précédent.
Dans ce TP, nous allons explorer une autre structure conditionnelle : \texttt{switch case}.


\begin{UPSTIinfor}{La structure \texttt{switch case}}
	La structure \texttt{switch case} est utilisée pour sélectionner l'une parmi plusieurs options basées sur la valeur d'une expression.
	Elle est particulièrement utile lorsque vous avez de nombreuses conditions basées sur la même variable.
	Voici la syntaxe de base d'une structure \texttt{switch case} en C :
	\begin{lstlisting}[language=c]
switch (expression) {
    case valeur1:
        // Code à exécuter si expression == valeur1
        break;
    case valeur2:
        // Code à exécuter si expression == valeur2
        break;
    // Vous pouvez avoir autant de cases que nécessaire
    default:
        // Code à exécuter si aucune des valeurs ne correspond
}
\end{lstlisting}
	\begin{itemize}
		\item \texttt{expression} est évaluée une fois et comparée aux valeurs dans chaque \texttt{case}.
		\item Si une correspondance est trouvée, le code associé à ce \texttt{case} est exécuté.
		\item Le mot-clé \texttt{break} est utilisé pour sortir de la structure \texttt{switch} après l'exécution d'un \texttt{case}. Sans \texttt{break}, l'exécution continue dans les \texttt{case} suivants (ce comportement est appelé "fall-through").
		\item Le \texttt{default} est optionnel et s'exécute si aucune des valeurs ne correspond.
	\end{itemize}
\end{UPSTIinfor}

\subsection{Exercice 0 : Exemple de \texttt{switch case}}
\begin{UPSTIManipulation}{Jours de la semaine}
    Compléter l'exemple suivant de \texttt{switch case} pour afficher le jour de la semaine en fonction d'un numéro (1 pour lundi, 2 pour mardi, etc.) :

\begin{lstlisting}[language=c]
#include <cs50.h>
#include <stdio.h>
int main() {
    int jour; // Variable pour stocker le numéro du jour
    jour = get_int("Entrez un numéro de jour (1-7) : ");

    switch (jour) {
        case 1:
            printf("Lundi\n");
            break;
        case 2:
            // Compléter pour les autres jours
        default:
            printf("Numéro de jour invalide. Veuillez entrer un nombre entre 1 et 7.\n");
    }
}
\end{lstlisting}
\tcblower 
\begin{lstlisting}
check50 IUT-GEII-Annecy/exercices/2025/info1/tp2/semaine
\end{lstlisting}
\end{UPSTIManipulation}

\begin{UPSTIinfor}{Le type \texttt{char}}
	Dans cet exercice, vous aurez besoin de demander un caractère simple. Le type le plus adapté est le type \texttt{char}.
	C'est un type qui peut contenir un caractère codé sur 1 octet (ASCII)

	\textbf{Attention} : En langage C, un caractère s'entoure d'apostrophes \texttt{' '} et non avec des guillements.
	\textit{Exemple :} \texttt{char caractere = 'A'}
\end{UPSTIinfor}