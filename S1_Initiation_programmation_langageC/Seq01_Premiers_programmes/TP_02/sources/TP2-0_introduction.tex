


\section{Convertisseur de température - Echauffement !}
On souhaite créer un programme permettant de convertir une température en degrés Celsius en degrés Fahrenheit et vice versa.

On donne la formule de conversion suivante :
\begin{itemize}
	\item Pour convertir des degrés Celsius (C) en degrés Fahrenheit (F) : F = C * 9/5 + 32
\end{itemize}

Donner la formule de conversion inverse (Fahrenheit vers Celsius).

\begin{UPSTIManipulation}{Conversion de température}
	Le programme affichera un menu permettant à l'utilisateur de choisir entre deux options :
	\begin{itemize}
		\item F : Convertir Celsius en Fahrenheit
		\item C : Convertir Fahrenheit en Celsius
	\end{itemize}
	L'utilisateur entrera son choix (F ou C) puis appuiera sur Entrée. Ensuite, le programme demandera la température à convertir.
	Le programme effectuera la conversion et affichera le résultat.\\
	\textbf{Attention :} Le choix de l'opération doit être fait avec un \texttt{switch case}. Structure \texttt{if} interdite.\\
	La dernière sortie doit être de la forme :
	`\texttt{<temperature\_utilisateur>} \texttt{<unite\_utilisateur>} = \texttt{<temperature\_convertie>} \texttt{<unite\_convertie>}\\
	Exemple d'exécution :
	\begin{lstlisting}[language=bash,style=console]
Choisissez l'opération (F pour Celsius->Fahrenheit, C pour Fahrenheit->Celsius) :
F 
Entrez la température en degrés Celsius :
25
25.00 C = 77.00 F
\end{lstlisting}
\tcblower
\begin{lstlisting}[language=bash]
check50 IUT-GEII-Annecy/exercices/2025/info1/tp2/temperatures
\end{lstlisting}
\end{UPSTIManipulation}


\section{Calculatrice - Entrainement !}
On souhaite créer une calculatrice basique qui effectue les opérations addition, soustraction, multiplication et division.


\subsection{Niveau 1 : Opérations}
\begin{UPSTIManipulation}{Calculatrice}
	Dossier : 2\_calculatrice\\
	Écrire un programme qui demande à l'utilisateur un nombre, une opération (addition, soustraction, multiplication, division), puis un second nombre. \textbf{L'utilisateur appuiera sur Entrée après chaque nombre/opération}.\\
	Pour choisir l'opération, l'utilisateur entrera \texttt{+}, \texttt{-}, \texttt{*}ou \texttt{/} respectivement.\\
	Le programme affichera alors l'opération et son résultat sur une ligne

	\begin{center}
		\begin{tabular}{|l|l|l|l|}
			\hline
			Entrée 1      & Entrée 2             & Entrée 3      & Sortie                                      \\
			\hline
			\texttt{<n1>} & \texttt{<operation>} & \texttt{<n2>} & \texttt{<n1> <operation> <n2> = <resultat>} \\
			\hline
		\end{tabular}
	\end{center}
	Avec \texttt{<n1>} et \texttt{<n2>} les deux nombres entrés par l'utilisateur, \texttt{<operation>} l'opération choisie et \texttt{<resultat>} le résultat de l'opération.
	\textbf{Attention :} Le choix de l'opéaration doit être fait avec un \texttt{switch case}. Structure \texttt{if} interdite.
	Exemple :
	\begin{lstlisting}[language=bash,style=console]
Bonjour, entrez votre opération : 
3
* 
2
Merci. Voici le résultat de votre opération : 
3 * 2 = 6
\end{lstlisting}
\tcblower
	check50 IUT-GEII-Annecy/exercices/2025/info1/tp2/calculatrice/niveau1
\end{UPSTIManipulation}

\subsection{Niveau 2 : Gestion des erreurs}

\begin{UPSTIManipulation}{Gestion des erreurs}
	\begin{itemize}
		\item[$\Box$] Ajouter la gestion des erreurs dans le programme de la calculatrice
		      \begin{itemize}
			      \item Si l'utilisateur entre une opération invalide, le programme doit afficher \texttt{ERREUR : Opération invalide} et terminer.
			      \item Si l'utilisateur entre une division par zéro, le programme doit afficher \texttt{ERREUR : Division par zéro} et terminer.
		      \end{itemize}
	\end{itemize}
	L'utilisation de \texttt{if} est autorisé pour cette partie.
	\begin{itemize}
		\item[$\Box$] Après avoir vérifié manuellement votre programme, faire valider par l'enseignant.
	\end{itemize}

	\tcblower 
\begin{lstlisting}
check50 IUT-GEII-Annecy/exercices/2025/info1/tp2/calculatrice/niveau2
\end{lstlisting}
\end{UPSTIManipulation}








