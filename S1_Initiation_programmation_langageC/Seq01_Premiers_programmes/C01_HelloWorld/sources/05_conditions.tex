
\section{Les conditions en C}

En C, les conditions permettent d'exécuter certaines parties du programme uniquement si des critères sont vérifiés.

\subsection{Structure générale : if / else}



\begin{lstlisting}[language=c]
if (<condition1>) // Si la condition 1 est vraie
{
    // Partie exécutée si la <condition1> est vérifiée
}
else if (<condition2>) // Sinon, si la condition 2 est vraie
{
    // Partie exécutée si la <condition2> est vérifiée
}
else // Sinon, dans tous les autres cas
{
    // Partie exécutée par défaut
}
\end{lstlisting}

La structure \texttt{if} peut prendre plusieurs formes :

\begin{itemize}
	\item \texttt{if} seul
	\item \texttt{if} puis \texttt{else}
	\item \texttt{if} puis \texttt{else if} puis \texttt{else}
\end{itemize}

On peut placer autant de \texttt{else if} que l'on veut à la suite.
\subsection{Opérateurs de comparaison}

Les comparaisons sont faites à l'aide des opérateurs suivants :

\begin{center}
	\begin{tabular}{|l|l|l|l|}
		\hline
		Symbole     & Signification     & Exemple (\texttt{x=5}, \texttt{y=10}) & Résultat \\
		\hline
		\texttt{==} & égal              & \texttt{x == 5}                       & vrai     \\
		\texttt{!=} & différent         & \texttt{x != y}                       & vrai     \\
		\texttt{<}  & inférieur         & \texttt{x < y}                        & vrai     \\
		\texttt{>}  & supérieur         & \texttt{x > y}                        & faux     \\
		\texttt{<=} & inférieur ou égal & \texttt{x <= 5}                       & vrai     \\
		\texttt{>=} & supérieur ou égal & \texttt{y >= 11}                      & faux     \\
		\hline
	\end{tabular}
\end{center}
\begin{UPSTIwarning}{Piège ! Ne pas confondre affectation et comparaison}
	L'opérateur \texttt{=} sert à affecter une variable et l'opérateur \texttt{==} sert à comparer deux variables ! Si vous les confondez, rien ne fonctionnera...
\end{UPSTIwarning}

\subsection{Opérateurs logiques}

On peut combiner plusieurs conditions avec les opérateurs logiques :

\begin{center}
	\begin{tabular}{|l|l|l|l|}
		\hline
		Symbole                                         & Signification & Exemple (x=5, y=10)                                            & Résultat attendu \\
		\hline
		\texttt{\&\&}                                   & ET logique    & \texttt{(x > 0 \&\& y < 20)}                                   & vrai             \\
		\texttt{\textbar\textbar} & OU logique    & \texttt{(x < 0 \textbar\textbar y > 100)} & vrai             \\
		\hline
	\end{tabular}
\end{center}
\begin{lstlisting}[language=c]
if ((x > 0) && (x < 10))
{
    printf("x est compris entre 0 et 10\n");
}

if ((x < 0) || (x > 100))
{
    printf("x est en dehors de l'intervalle [0,100]\n");
}
\end{lstlisting}

\subsection{Un exemple simple}

Si \texttt{x} est une variable entière, le code suivant vérifie si \texttt{x} est positif, négatif ou nul.

\begin{lstlisting}[language=c]
if (x > 0)
{
    printf("x est positif\n");
}
else if (x < 0)
{
    printf("x est négatif\n");
}
else
{
    printf("x vaut zéro\n");
}
\end{lstlisting}