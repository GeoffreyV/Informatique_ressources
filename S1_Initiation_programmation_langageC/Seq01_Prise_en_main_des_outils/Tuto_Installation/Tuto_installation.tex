\documentclass[TP, noCustomPackages]{UPSTI_Document}
\usepackage{IUT_Annecy}
\usepackage{logos_iut}

\newcommand{\sequence}{2}
\newcommand{\UPSTIsousTitreEnTete}{Sequence \sequence~: Boucles et fonctions}
\newcommand{\UPSTItitre}{Boucles et premières fonctions}

\graphicspath{{images/}}


\documentVersion{E}
\newcommand{\UPSTInumeroVersion}{1}
\newcommand{\UPSTInumero}{\sequence.2}
\begin{document}

\UPSTIobjectif{Installer un environnement de développement pour le langage C}

\section*{Installation de Visual Studio Code et des outils pour le C}

Pour programmer en C, il est essentiel d’installer un éditeur de texte performant, un compilateur, un terminal, et des outils de débogage. Ce tutoriel vous guide dans l’installation d’un environnement complet sur votre ordinateur personnel.

\begin{UPSTIinfor}{Visual Studio Code}
\textbf{Visual Studio Code} (VS Code) est un éditeur de texte léger, rapide et très modulable grâce à ses extensions.  
Il est gratuit et disponible pour Windows, macOS et Linux.

\begin{itemize}
    \item Télécharger VS Code : \href{https://code.visualstudio.com/}{\texttt{https://code.visualstudio.com/}}
    \item Installer en suivant les instructions selon votre système d’exploitation.
\end{itemize}

Une fois installé, lancez VS Code.
\end{UPSTIinfor}

\begin{UPSTIinfor}{Extensions à installer dans VS Code}

Ouvrez l’onglet des extensions (icône à gauche ou \texttt{Ctrl+Shift+X}) et installez les extensions suivantes :
\begin{itemize}
    \item \texttt{C/C++} par Microsoft — pour le support du langage C (intelliSense, debug, etc.).
    \item \texttt{Code Runner} (optionnel) — pour exécuter rapidement des programmes simples.
    \item \texttt{GitLens} (optionnel) — pour mieux visualiser l’historique Git dans l’éditeur.
\end{itemize}

\end{UPSTIinfor}

\begin{UPSTIinfor}{Installation du compilateur C}

Pour compiler du code C, vous devez installer le compilateur \texttt{gcc}. Voici comment faire selon votre système.

\textbf{Sous Windows (avec MinGW)} :
\begin{itemize}
    \item Télécharger l’installateur : \href{https://www.mingw-w64.org/}{\texttt{https://www.mingw-w64.org/}}
    \item Lors de l’installation, choisissez l’architecture \texttt{x86\_64}, le mode POSIX, et l’exception \texttt{seh}.
    \item Ajouter le chemin de \texttt{bin/} (par ex. \texttt{C:\textbackslash Program Files\textbackslash mingw-w64\textbackslash ...\textbackslash bin}) à la variable d’environnement \texttt{PATH}.
    \item Vérifier l’installation dans un terminal :\\
    \texttt{gcc --version}
\end{itemize}

\textbf{Sous macOS} :
\begin{itemize}
    \item Installer les outils de développement :\\
    \texttt{xcode-select --install}
    \item Vérifier avec : \texttt{gcc --version}
\end{itemize}

\textbf{Sous Linux (Debian/Ubuntu)} :
\begin{itemize}
    \item Dans un terminal, exécuter :\\
    \texttt{sudo apt update} \\
    \texttt{sudo apt install build-essential}
\end{itemize}

\end{UPSTIinfor}

\begin{UPSTIinfor}{Terminal intégré}

Vous pouvez utiliser le terminal intégré de VS Code :
\begin{itemize}
    \item Ouvrir avec \texttt{Ctrl + \`}, ou via le menu \texttt{Terminal > Nouveau terminal}.
    \item Utiliser les commandes \texttt{gcc}, \texttt{make}, etc.
\end{itemize}

\end{UPSTIinfor}

\begin{UPSTIinfor}{Débogueur GDB}

Pour pouvoir déboguer des programmes :
\begin{itemize}
    \item Sous Windows : vérifier que GDB est inclus dans MinGW (sinon l’installer séparément).
    \item Sous Linux ou macOS : généralement installé par défaut avec \texttt{gcc}, sinon installer avec \texttt{apt} ou \texttt{brew}.
\end{itemize}
Configurer le débogueur dans VS Code en suivant le guide fourni par l’extension \texttt{C/C++}.

\end{UPSTIinfor}

\begin{UPSTIinfor}{Git pour la gestion de versions}

\begin{itemize}
    \item Télécharger Git : \href{https://git-scm.com/}{\texttt{https://git-scm.com/}}
    \item Suivre les étapes d’installation (cocher "Utiliser Git depuis l'invite de commande").
    \item Vérifier dans un terminal :\\
    \texttt{git --version}
\end{itemize}

VS Code détecte Git automatiquement. Vous pourrez gérer vos projets avec des commits, branches et dépôts GitHub.

\end{UPSTIinfor}

\begin{UPSTIinfor}{À l'IUT d'Annecy}

Les ordinateurs de l'IUT sont déjà configurés avec tous ces outils.  
Néanmoins, vous êtes encouragés à reproduire cet environnement chez vous pour vous entraîner et travailler à distance.

Un récapitulatif de ces étapes est disponible sur le GitHub du module.

\end{UPSTIinfor}

\end{document}
