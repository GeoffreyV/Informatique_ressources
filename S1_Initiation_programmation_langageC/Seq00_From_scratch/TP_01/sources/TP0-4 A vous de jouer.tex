\section{A vous de jouer !}
A vous de créer un programme Scratch qui vous ressemble et qui montre vos compétences en programmation.

Ce premier rendu peut être un jeu, une animation, une histoire interactive, ou tout autre projet qui vous inspire. Les seules contraintes sont : 
\begin{UPSTIaRendre}{Projet Scratch}    
    \begin{itemize}
    \item Votre projet doit utiliser au moins deux lutins (sprites), dont au moins un ne doit pas être un chat.
    \item Votre projet doit comporter au moins trois scripts au total (c'est-à-dire pas nécessairement trois par lutin).
    \item Votre projet doit utiliser au moins une conditionnelle, au moins une boucle et au moins une variable.
    \item Votre projet doit utiliser au moins un bloc personnalisé que vous avez créé vous-même (via Créer un bloc), qui doit prendre au moins une entrée.
    \item Votre projet doit être interactif, c'est-à-dire que l'utilisateur doit pouvoir interagir avec le programme d'une manière ou d'une autre (par exemple, en cliquant sur un sprite, en appuyant sur une touche, etc.).
    \end{itemize}
\end{UPSTIaRendre}

Le rendu se fera sur la plateforme moodle, où vous devrez soumettre le lien vers votre projet Scratch.  