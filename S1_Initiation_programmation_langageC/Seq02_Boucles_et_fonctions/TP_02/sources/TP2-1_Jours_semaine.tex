
\section{Validateur de date}

\subsection{Niveau 1 — Valider une date (jour/mois/année)}

\begin{UPSTIManipulation}{Validateur de date}
	\textbf{Dossier :} \texttt{3\_dates}
	Écrire un programme qui lit trois entiers dans cet ordre : \textbf{jour}, \textbf{mois}, \textbf{année} (l’utilisateur appuie sur Entrée après chaque saisie).\\
	Le programme doit \textbf{vérifier} si la date est \textbf{valide} en tenant compte des différentes longueurs de mois \textbf{et} des \textbf{années bissextiles}.

	\textbf{Contraintes / attentes :}
	\begin{itemize}
		\item Utiliser un \textbf{\texttt{switch (mois)}} pour déterminer \textbf{le nombre de jours} dans le mois (\texttt{maxJours}).
		\item Penser aux cas particuliers :
		      \begin{itemize}
			      \item Le \texttt{mois} doit être valide
			      \item Le \texttt{jour} doit exister (attention aux années bixestiles)
			      \item
		      \end{itemize}
		\item \textbf{Affichages attendus :}
		      \begin{itemize}
			      \item Si la date est valide : \texttt{Date valide}
			      \item Sinon : \texttt{ERREUR : Date invalide.}
		      \end{itemize}
	\end{itemize}
	\textbf{Rappels utiles :}
	\begin{itemize}
		\item \textbf{Année bissextile} :
	\end{itemize}
	Une année est bissextile si l'une de ces deux conditions est remplie :
	\begin{itemize}
		\item L'année est un mutiple de 400
		\item L'année est un multiple de 4 mais pas un multiple de 100.
	\end{itemize}
	\texttt{(année \% 400 == 0) || (année \% 4 == 0 \&\& année \% 100 != 0)}
	\begin{itemize}
		\item \textbf{Longueur des mois} (hors février) :
		      \begin{itemize}
			      \item 31 jours : 1, 3, 5, 7, 8, 10, 12
			      \item 30 jours : 4, 6, 9, 11
		      \end{itemize}
	\end{itemize}
	\textbf{Exemples d’exécution :}
	\begin{lstlisting}[language=bash]
Entrez le jour :
29
Entrez le mois :
2
Entrez l'année :
2024
Date valide
\end{lstlisting}
	\begin{lstlisting}[language=bash]
Entrez le jour :
31
Entrez le mois :
4
Entrez l'année :
2025
ERREUR : Date invalide.
\end{lstlisting}
	\textbf{Types conseillés :} \texttt{int} pour jour, mois, année.
	\textbf{Pas de boucles ni de tableaux} nécessaires pour ce niveau.
	\tcblower
	\begin{lstlisting}[language=bash]
check50 IUT-GEII-Annecy/exercices/2025/info1/tp2/date/niveau1
	\end{lstlisting}
\end{UPSTIManipulation}

\begin{UPSTIinfor}{Structure possible (pseudo‑code)}
	1) Lire \texttt{jour}, \texttt{mois}, \texttt{annee}
	2) \texttt{Si mois n'est pas entre 1 et 12} → invalide
	3) \texttt{switch (mois)} → donner \texttt{maxJours} (28 par défaut pour février)
	4) \texttt{Si l'année est bissextile} → \texttt{maxJours = 29}
	5) \texttt{Si le jour est plus grand que maxJours (ou inférieur à 1)} → invalide, sinon valide
\end{UPSTIinfor}
\hrule
\subsection{Niveau 2 — Ajouter la saison (météorologique)}

\begin{UPSTIManipulation}{Saison (en plus de la validation)}
	Reprendre le programme du \textbf{Niveau 1}. \\
	S’il détecte une \textbf{date valide}, il doit \textbf{afficher} la \textbf{saison météorologique} correspondante en \textbf{France} :
	\begin{itemize}
		\item \textbf{Printemps} : du \textbf{21 mars} au \textbf{20 juin}
		\item \textbf{Été} : du \textbf{21 juin} au \textbf{20 septembre}
		\item \textbf{Automne} : du \textbf{21 septembre} au \textbf{20 décembre}
		\item \textbf{Hiver} : du \textbf{21 décembre} au \textbf{20 mars}
	\end{itemize}

	\begin{itemize}
		\item \textbf{Affichages attendus :}
		      \begin{itemize}
			      \item Si la date est valide :
		      \end{itemize}
	\end{itemize}
	\texttt{Date valide} (à la ligne suivante) \texttt{Saison : <saison>}
	\begin{itemize}
		\item Sinon : \texttt{ERREUR : Date invalide}
	\end{itemize}
	\textbf{Exemples d’exécution :}
	\begin{lstlisting}[language=bash]
Entrez le jour :
15
Entrez le mois :
6
Entrez l'année :
2025
Date valide
Saison : été
\end{lstlisting}
	\begin{lstlisting}[language=bash]
Entrez le jour :
1
Entrez le mois :
12
Entrez l'année :
2025
Date valide
Saison : Hiver
\end{lstlisting}
	\tcblower
	\begin{lstlisting}[language=bash]
check50 IUT-GEII-Annecy/exercices/2025/info1/tp2/date/niveau2
	\end{lstlisting}
\end{UPSTIManipulation}