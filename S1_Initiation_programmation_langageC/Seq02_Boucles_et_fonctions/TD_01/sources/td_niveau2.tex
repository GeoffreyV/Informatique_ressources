\section{Niveau 2}
\begin{UPSTIexercice}{Dessiner un rectangle avec une fonction}
    \UPSTIquestion{Ecrire une fonction \texttt{dessiner\_ligne(int longueur)} qui dessine une ligne d'étoiles de longueur `longueur`.}
    \UPSTIquestion{Ecrire un programme qui utilise cette fonction pour dessinner un rectangle d'étoiles à partir des nombres hauteur et largeur demandés à l'utilisateur.}
    \paragraph{Exemple, pour hauteur=2 et largeur=4 }
     \begin{lstlisting}
****
****
     \end{lstlisting}
     \UPSTIquestion{Comparer cette solution avec celle du niveau 1.}
\end{UPSTIexercice}



\begin{UPSTIexercice}{Moyenne de n nombres}
  \UPSTIquestion{Ecrire un programme qui calcule la moyenne de `n` nombres entrés par l'utilisateur. \textit{Le programme demandera d'abord combien de nombres l'utilisateur veut utiliser avant de demander les nombres en question.}}
\end{UPSTIexercice}

\begin{UPSTIprofOnlyEnv}
\begin{UPSTIcorrectionP}{Moyenne}
    \lstinputlisting{sources/moyenne.c}
\end{UPSTIcorrectionP}
\end{UPSTIprofOnlyEnv}

\begin{UPSTIexercice}{Intérêts bancaire}
  \UPSTIquestion{Ecrire un programme qui calcule le montant d'un placement après `n` années, sachant que le taux d'intérêt annuel est de 3.5\%. Le programme demandera à l'utilisateur le montant initial du placement et le nombre d'années `n`.}
  \UPSTIquestion{Ecrire une une version qui calcule le nombre d'années nécessaires pour que le placement double.}  
\end{UPSTIexercice}

\begin{UPSTIprofOnlyEnv}
\begin{UPSTIcorrectionP}{Intérêts bancaire}
    \lstinputlisting{sources/interets_bancaire.c}
\end{UPSTIcorrectionP}
\end{UPSTIprofOnlyEnv}

\begin{UPSTIexercice}{Compter les moutons et les poulets}
  Un fermier fait l'élevage de poulets et de moutons. Un jour, un voleur entre dans la ferme et vole un certain nombre d'animaux. Le fermier constate qu'il reste 74 pattes.
  \UPSTIquestion{Ecrire un programme qui écrit toutes les combinaisons possibles de poulets et de moutons qu'il peut y avoir dans la ferme.}
  \textit{Vous utiliserez une boucle pour tester toutes les combinaisons possibles.} 
\end{UPSTIexercice}

\begin{UPSTIprofOnlyEnv}
\begin{UPSTIcorrectionP}{Compter les moutons et les poulets}
    \lstinputlisting{sources/poulets_moutons.c}
\end{UPSTIcorrectionP}
\end{UPSTIprofOnlyEnv}

 \begin{UPSTIexercice}{Prédire l'affichage de boucles}
   \UPSTIquestion{Prédire l'affichage des programmes suivants.}
   \begin{lstlisting}
#include <stdio.h>
int main() {
    int i;
    for (i=0; i<5; i++) {
        printf("*");
    }
    return 0;
}
   \end{lstlisting}
   \begin{lstlisting}
#include <stdio.h>
int main() {
    int i;
    for (i=0; i<5; i++) {
        printf("*\n");
    }
    return 0;
}
   \end{lstlisting}
   \begin{lstlisting}
#include <stdio.h>
int main() {
    int i;
    int a = 0, b = 0;
    for (i=0; i<3; i++) {
            a = a + 1;
            b = b + 2;
            printf("a=%i b=%i\n", a, b);
    }
    return 0;
}
   \end{lstlisting}
   \begin{lstlisting}
#include <stdio.h>
int main() {
    int i;  
    for (i=0; i<4; i++) {
        if (i%2 == 0) {
            printf("*");
        } else {
            printf("#");
        }
    }
    return 0;
}
   \end{lstlisting}
 \end{UPSTIexercice}