\section{Niveau 3}

\begin{UPSTIexercice}{Nombres premiers}
  \UPSTIquestion{Ecrire un programme qui demande un entier `n` à l'utilisateur, puis qui indique si \texttt{n} est un nombre premier ou non. S'il ne l'est pas, le programme affichera les diviseurs de \texttt{n}}
\textit{Remarque : un nombre premier est un entier naturel supérieur à 1 qui n'a que deux diviseurs distincts : 1 et lui-même.}  
\end{UPSTIexercice}

\begin{UPSTIprofOnlyEnv}
\begin{UPSTIcorrectionP}{Nombres premiers}
    \lstinputlisting{sources/nombres_premiers.c}
\end{UPSTIcorrectionP}
\end{UPSTIprofOnlyEnv}

\begin{UPSTIexercice}{Puissances de 2}
  \UPSTIquestion{Ecrire un programme qui affiche les puissances de 2 inférieures ou égales à un entier `n` donné par l'utilisateur.}
  \UPSTIquestion{Ecrire une version qui utilise une fonction \texttt{puissance\_de\_2(int n)} qui renvoie la plus grande puissance de 2 inférieure ou égale à `n`.}
\end{UPSTIexercice}

\begin{UPSTIprofOnlyEnv}
\begin{UPSTIcorrectionP}{Puissances de 2}
    \lstinputlisting{sources/puissances_de_2.c}
\end{UPSTIcorrectionP}
\end{UPSTIprofOnlyEnv}

