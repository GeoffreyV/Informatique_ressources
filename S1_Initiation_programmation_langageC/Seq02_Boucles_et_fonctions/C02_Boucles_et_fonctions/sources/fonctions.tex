
\section{Les fonctions void}

Cette section se concentre sur les fonctions `void`, fontions qui produisent un effet sans retourner de valeur. Les fonctions d'affichages en sont des exemples typiques.

\begin{UPSTIinfor}{Une fonction}
	Une fonction sert à regrouper un ensemble d'instructions que l'on peut réutiliser plusieurs fois dans un programme. Cela permet de rendre le code plus lisible, plus modulaire et plus facile à maintenir.
\end{UPSTIinfor}

\subsection{Définir une fonction}
Pour définir une fonction en C, on utilise la syntaxe suivante :

\begin{lstlisting}[language=c]
type_de_retour nom_de_fonction(paramètres) {
    // bloc d'instructions
}
\end{lstlisting}

\begin{itemize}
	\item \textbf{type\_de\_retour} : C'est le type de donnée que la fonction renvoie après son exécution (par exemple, \texttt{int}, \texttt{float}, \texttt{char}, etc.). Si la fonction ne renvoie rien, on utilise le mot clef \texttt{void}.
	\item \textbf{nom\_de\_fonction} : C'est le nom que vous donnez à la fonction. Il doit être descriptif de ce que fait la fonction.
	\item \textbf{paramètres} : Ce sont les variables que la fonction peut prendre en entrée pour effectuer son travail. Elles sont optionnelles. Si la fonction ne prend pas de paramètres, on laisse les parenthèses vides.
	\item \textbf{bloc d'instructions} : C'est le code qui sera exécuté lorsque la fonction est appelée.
\end{itemize}

\subsubsection{Exemple rapide : Une fonction void qui affiche un message de bienvenue}

\begin{lstlisting}[language=c]
#include <stdio.h>

void afficher_bienvenue(); // Déclaration de la fonction

int main(void) {
    afficher_bienvenue(); // Appel de la fonction
    return 0;
}

void afficher_bienvenue() { // Définition de la fonction
    printf("Bienvenue dans le monde de la programmation en C !\n");
}
\end{lstlisting}

Trois choses importantes à noter dans cet exemple :
\begin{enumerate}
	\item La fonction \texttt{afficher\_bienvenue} est déclarée avant son utilisation dans \texttt{main}. Cela informe le compilateur qu'une fonction avec ce nom et cette signature existe.
	\item La fonction est définie après \texttt{main} -- mais elle pourrait aussi être définie avant.
	\item La fonction est \textbf{appelée} dans \texttt{main} en utilisant son nom suivi de parenthèses.
\end{enumerate}

\subsubsection{Exemple du cours : Fonction qui affiche une ligne de \texttt{n} blocs}
