\section{Boucles en C — while, for, do...while}

\begin{UPSTIinfor}{Objectifs}
	\begin{itemize}
		\item Comprendre le fonctionnement de la boucle concernée.
		\item Savoir \textbf{quand} l'utiliser par rapport aux alternatives.
		\item Éviter les boucles infinies et apprendre à \textbf{déboguer}.
	\end{itemize}
\end{UPSTIinfor}

\subsection{Une boucle, c'est quoi ?}
Les boucles permettent de \textbf{répéter} un bloc d'instructions tant qu'une condition est vraie
(ou un compteur pas encore à terme).

\subsection{Exemple rapide : Miauler 3 fois avec \texttt{while}}

\begin{lstlisting}[language=c]
#include <stdio.h>
int main(void) {
    int i = 0;
    while (i < 3) {
        printf("Miaou !\n");
        i++;
    }
    return 0;
}
\end{lstlisting}
Ici, tant que \texttt{i} est inférieur à 3, on affiche "Miaou !" et on incrémente \texttt{i} de 1.
Cela produit l'affichage suivant :
\begin{lstlisting}[language=bash,style=console]
Miaou !
Miaou !
Miaou ! 
\end{lstlisting}


\begin{lstlisting}[language=c]
#include <stdio.h>

int main(void) {
    int i = 0;
    while (i < 3) {
        printf("i = %d\n", i);
        i++;
    }
    return 0;
}
\end{lstlisting}



\begin{UPSTIwarning}{Pièges}
	\begin{itemize}
		\item Condition mal mise à jour -> \textbf{boucle infinie}.
		\item Off‑by‑one (\texttt{<} vs \texttt{<=}).
		\item Variables modifiées \textbf{dans} la condition.
	\end{itemize}
\end{UPSTIwarning}

\subsection{Exercices flash}
\begin{enumerate}
	\item Afficher les nombres de 1 à 10 sur une ligne.
	\item Somme des entiers de 1 à \texttt{n} (entré par l'utilisateur).
	\item Lire jusqu'à ce que l'utilisateur tape \texttt{0}, puis afficher le comptage.
\end{enumerate}