
\begin{UPSTIManipulation}{Table de multiplication}
	\begin{itemize}
		\item[$\Box$] Écrire un programme \texttt{table\_multiplication.c} qui demande à l'utilisateur un entier \texttt{n} et affiche la table de multiplication de \texttt{n} de 1 à 10 en utilisant une boucle \texttt{for}.
	\end{itemize}
	Par exemple, si l'utilisateur entre \texttt{5}, le programme affichera :
	\begin{lstlisting}[language=bash,style=console]
5 x 1 = 5
5 x 2 = 10
5 x 3 = 15
5 x 4 = 20
5 x 5 = 25
5 x 6 = 30
5 x 7 = 35
5 x 8 = 40
5 x 9 = 45
5 x 10 = 50
\end{lstlisting}
	\hrule
	check50 IUT-GEII-Annecy/squelettes/branch-2025/tp3/maths/table\_multiplication
\end{UPSTIManipulation}

\begin{UPSTIManipulation}{Somme des n premiers entiers}
	\begin{itemize}
		\item[$\Box$] Écrire un programme \texttt{somme\_n\_entiers.c} qui demande à l'utilisateur un entier \texttt{n} et calcule la somme des \texttt{n} premiers entiers (de 1 à \texttt{n}) en utilisant une boucle \texttt{for}. Le programme doit ensuite afficher le résultat.
	\end{itemize}
	Par exemple, si l'utilisateur entre \texttt{5}, le programme affichera :
	\begin{lstlisting}[language=bash,style=console]
1 + 2 + 3 + 4 + 5 = 15
\end{lstlisting}
	\textbf{Utiliser deux boucles différentes pour le calcul de la somme et pour l'affichage du résultat.}
	\hrule
	check50 IUT-GEII-Annecy/squelettes/branch-2025/tp3/maths/somme\_n\_entiers
\end{UPSTIManipulation}

\begin{UPSTIManipulation}{Factorielle}
	\begin{itemize}
		\item[$\Box$] Écrire un programme \texttt{factorielle.c} qui demande à l’utilisateur un entier \texttt{n} et calcule \texttt{n!} (la factorielle de \texttt{n}) en utilisant une boucle \texttt{for}.
	\end{itemize}
	Par exemple, pour \texttt{n = 5}, le programme affichera :
	\begin{lstlisting}[language=bash,style=console]
5! = 1 x 2 x 3 x 4 x 5 = 120
\end{lstlisting}
	\hrule
	check50 IUT-GEII-Annecy/squelettes/branch-2025/tp3/maths/factorielle
\end{UPSTIManipulation}

\begin{UPSTIManipulation}{Carre}
	\begin{itemize}
		\item[$\Box$] Écrire un programme \texttt{carre.c} qui demande à l'utilisateur un entier \texttt{n} et affiche un carré de \texttt{n} par \texttt{n} utilisant des astérisques (\texttt{*}). Utiliser une boucle \texttt{for} imbriquée pour générer les lignes et les colonnes du carré.
	\end{itemize}
	Par exemple, si l'utilisateur entre \texttt{4}, le programme affichera :
	\begin{lstlisting}[language=bash,style=console]
****
****
****
****
\end{lstlisting}
\end{UPSTIManipulation}

\begin{UPSTIManipulation}{Triangle -- Niveau 1}
	\begin{itemize}
		\item[$\Box$] Écrire un programme \texttt{triangle1.c} qui demande à l’utilisateur un entier \texttt{n} et affiche un triangle d’étoiles aligné à gauche, de hauteur \texttt{n}.
	\end{itemize}
	Par exemple, pour \texttt{n = 4} :
	\begin{lstlisting}[language=bash,style=console]
*
**
***
****
\end{lstlisting}
	\hrule
	check50 IUT-GEII-Annecy/squelettes/branch-2025/tp3/formes/triangle/niveau1
\end{UPSTIManipulation}

\begin{UPSTIManipulation}{Triangle -- Niveau 2}
	\begin{itemize}
		\item[$\Box$] Écrire un programme \texttt{triangle2.c} qui demande à l’utilisateur un entier \texttt{n} et affiche un triangle isocèle aligné au centre, de hauteur \texttt{n}.
	\end{itemize}
	Par exemple, pour \texttt{n = 4} :
	\begin{lstlisting}[language=bash,style=console]
   *
  ***
 *****
*******    
\end{lstlisting}
	\hrule
	check50 IUT-GEII-Annecy/squelettes/branch-2025/tp3/formes/triangle/niveau2
\end{UPSTIManipulation}

\begin{UPSTIManipulation}{Damier}
	\begin{itemize}
		\item[$\Box$] Écrire un programme \texttt{damier.c} qui demande deux entiers \texttt{lignes} et \texttt{colonnes} et affiche un damier en alternant \texttt{\#} et \texttt{.}.
	\end{itemize}
	Par exemple, pour \texttt{lignes = 4} et \texttt{colonnes = 6} :
	\begin{lstlisting}[language=bash,style=console]
#.#.#.
.#.#.#
#.#.#.
.#.#.#
\end{lstlisting}
	\hrule
	check50 IUT-GEII-Annecy/squelettes/branch-2025/tp3/formes/damier
\end{UPSTIManipulation}

\begin{UPSTIManipulation}{Motif en diagonale}
	\begin{itemize}
		\item[$\Box$] Écrire un programme \texttt{diagonale.c} qui demande un entier \texttt{n} et affiche un carré de taille \texttt{n} avec une diagonale de \texttt{*}.
	\end{itemize}
	Exemple, pour \texttt{n = 5} :
	\begin{lstlisting}[language=bash,style=console]
*....
.*...
..*..
...*.
....*
\end{lstlisting}
	\hrule
	check50 IUT-GEII-Annecy/squelettes/branch-2025/tp3/formes/diagonale
\end{UPSTIManipulation}