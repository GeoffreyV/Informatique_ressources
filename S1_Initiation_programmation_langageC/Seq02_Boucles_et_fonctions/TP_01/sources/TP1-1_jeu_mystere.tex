
\begin{UPSTIManipulation}{Devine le nombre - Version While, avec n vies}
	Un nombre mystère est choisi au hasard entre 1 et 100. Le joueur a \texttt{n} vies pour deviner ce nombre.

	Tout d'abord, le programme affiche un menu avec 4 niveux de difficulté, qu'il choisira en entrant un chiffre entre 1 et 4 :
	\begin{enumerate}
		\item Facile (10 vies)
		\item Moyen (7 vies)
		\item Difficile (5 vies)
		\item Impossible (0 vies)
	\end{enumerate}
	Ensuite, le programme choisit un nombre au hasard entre 1 et 100, puis demande au joueur de deviner ce nombre.

	\begin{itemize}
		\item Après chaque proposition, le programme indique si le nombre mystère est plus grand ou plus petit que la proposition.
		\item Le jeu continue jusqu'à ce que le joueur trouve le nombre mystère ou qu'il n'ait plus de vies.
		\item Si le joueur trouve le nombre mystère, le programme affiche un message de félicitations.
		\item Si le joueur n'a plus de vies, le programme affiche un message de défaite et révèle le nombre mystère.
		\item Le programme doit utiliser une boucle \texttt{while} pour gérer les tentatives du joueur.
	\end{itemize}
	Le programme \texttt{devine\_le\_nombre\_while.c}contient le code suivant :
	\begin{lstlisting}[language=c]
#include <stdio.h>
#include <stdlib.h>
#include <time.h>
int main(void) {
    int vies;
    int nombre_mystere;
    int proposition;
    int niveau;
   // A COMPLETER : Demander le niveau de difficulté et initialiser le nombre de vies
   // L'utilisation d'un switch case parait appropriée ici. 
    // Initialiser le générateur de nombres aléatoires
    srand(time(NULL));
    nombre_mystere = rand() % 100 + 1; // Nombre entre 1 et 100
   // A COMPLETER : Boucle principale du jeu
   return 0;
}
\end{lstlisting}
	\begin{itemize}
		\item[$\Box$] Compléter le programme \texttt{devine\_le\_nombre\_while.c} pour qu'il fonctionne comme décrit.
		\item[$\Box$] Tester le programme avec chaque niveau de difficulté.
		\item[$\Box$] Que se passe-t-il si le joueur choisit le niveau "Impossible" ? Pourquoi ?
		\item[$\Box$] Que se passe-t-il si le joueur choisi un niveau invalide (par exemple 5) ? Pourquoi ?
	\end{itemize}
	\hrule
	check50 IUT-GEII-Annecy/squelettes/branch-2025/tp3/devinelenombre/while
\end{UPSTIManipulation}

\begin{UPSTIManipulation}{Devine le nombre - Version For, avec n vies}
	\begin{itemize}
		\item[$\Box$] Faire le même exercice que précédemment, mais en utilisant une boucle \texttt{for} pour gérer les tentatives du joueur.
	\end{itemize}
	\hrule
	check50 IUT-GEII-Annecy/squelettes/branch-2025/tp3/devinelenombre/for
\end{UPSTIManipulation}

\begin{UPSTIManipulation}{Devine le nombre - Version Do...While, avec n vies}
	\begin{itemize}
		\item[$\Box$] Faire le même exercice que précédemment, mais en utilisant une boucle \texttt{do...while} pour gérer les tentatives du joueur.
		\item[$\Box$] Tester le programme avec n = 0. Que constatez-vous ?
		\item[$\Box$] Ce type de boucle vous parait-il pertinent pour ce jeu ? Pourquoi ?
	\end{itemize}
	\hrule
	check50 IUT-GEII-Annecy/squelettes/branch-2025/tp3/devinelenombre/doWhile
\end{UPSTIManipulation}

\begin{UPSTIManipulation}{Vérifier le choix d'un menu}
	\begin{itemize}
		\item[$\Box$] Quelle boucle est la plus adaptée pour vérifier que l'utilisateur a bien entré un choix valide dans un menu ? Justifier.
		\item[$\Box$] Vérifier votre choix avec l'enseignant.
		\item[$\Box$] Modifier vos programme précédents pour utiliser la boucle choisie.
	\end{itemize}
\end{UPSTIManipulation}